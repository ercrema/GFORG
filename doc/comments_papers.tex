\documentclass{article}

\usepackage[utf8]{inputenc}
\usepackage[round]{natbib}
\usepackage{graphicx}
\usepackage{color}

\newcommand{\todo}[1]{
	\textcolor{blue}{\textbf{#1}}
}

\begin{document}

Some papers and comments:

\section{An evolutionary analysis of the volunteer's dilemma}

Myatt \& Wallace, 2008, Games and Economic Behavior, 62, 67-76

The authors note that noisy strategy evaluations are responsible for the evolution of the game. From what I understand, this is exactly the same thing as to implement a mutation mechanism. I'm not sure whether it is too artificial...

\section{Volunteer's Dilemma}

Diekmann, 1985, Journal of Conflict Resoliution, vol. 29, 4, 605-610

\emph{'For given group soze N, both [models] predict that the probability of defection is a monotonically increasing function of the cost-benefit ratio'}


\section{Compromise group size}

Higashi and Yamamusa, 1993, The American Naturalist, vol 142, 3, 553-563

Follows the literature using a single humped fitness curve and relatedness. 
When conflict arises, all members of the group pay the same cost $d$, so the total cost is $nd$ where $n$ is group size.
The joiner also pays a cost (we haven't considered this). If the joiner pays a cost $knd$ (where $k$ is the insider's advantage) then it gets in.
They then identify a \emph{compromise group size} where the two forces are in equilibrium. This is smaller than the equilibrium size (fitness equal to individual forager)
and larger than the optimal group size.



\end{document}

